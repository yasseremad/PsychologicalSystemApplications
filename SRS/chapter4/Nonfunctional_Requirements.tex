\documentclass[../Psychological_system_web_application.tex]{subfiles}

\begin{document}
	
	
		\section{Performance Requirements}
		
			\paragraph{our system has very fast performance We built it and design it to work very fast without any issues.}
		
		\section{Safety Requirements}
		
			\paragraph{our system has no effects or dangerous on the users or machines because it is software system.}
		
		\section{Security Requirements}
		
			\paragraph{we secure our system with the newest technology in security field to provide the best experience to the user without any fearing of security.}
		
		\section{Software Quality Attributes}
		
			\subsection{Availability}
				\paragraph{\gls{Availability} defines the proportion of time that the system is functional and working. It can be measured as a percentage of the total system downtime over a predefined period. Availability will be affected by system errors, infrastructure problems, malicious attacks, and system load. The key issues for availability are:}
				\begin{itemize}
					\item
						A physical tier such as the database server or application server can fail or become unresponsive, causing the entire system to fail. Consider how to design fail-over support for the tiers in the system. For example, use Network Load Balancing for Web servers to distribute the load and prevent requests being directed to a server that is down. Also, consider using a RAID mechanism to mitigate system failure in the event of a disk failure. Consider if there is a need for a geographically separate redundant site to fail-over to in case of natural disasters such as earthquakes or tornado.
					\item
						\acr{DoS} attacks, which prevent authorized users from accessing the system, can interrupt operations if the system cannot handle massive loads in a timely manner, often due to the processing time required, or network configuration and congestion. To minimize interruption from DoS attacks, reduce the attack surface area, identify malicious behavior, use application instrumentation to expose unintended behavior, and implement comprehensive data validation. Consider using the Circuit Breaker or Bulkhead patterns to increase system resiliency.
					\item
						Inappropriate use of resources can reduce availability. For example, resources acquired too early and held for too long cause resource starvation and an inability to handle additional concurrent user requests.
					\item
						Bugs or faults in the application can cause a system wide failure. Design for proper exception handling in order to reduce application failures from which it is difficult to recover.
					\item
						Frequent updates, such as security patches and user application upgrades, can reduce the availability of the system. Identify how you will design for run-time upgrades.
					\item
						A network fault can cause the application to be unavailable. Consider how you will handle unreliable network connections; for example, by designing clients with occasionally-connected capabilities.
					\item
						Consider the trust boundaries within your application and ensure that subsystems employ some form of access control or firewall, as well as extensive data validation, to increase resiliency and availability.
				\end{itemize}
			
			\subsection{Conceptual Integrity}
				\paragraph{\gls{Conceptual integrity} defines the consistency and coherence of the overall design. This includes the way that components or modules are designed, as well as factors such as coding style and variable naming. A coherent system is easier to maintain because you will know what is consistent with the overall design. Conversely, a system without conceptual integrity will constantly be affected by changing interfaces, frequently deprecating modules, and lack of consistency in how tasks are performed. The key issues for conceptual integrity are:}
				
				\begin{itemize}
					\item
						Mixing different areas of concern within your design. Consider identifying areas of concern and grouping them into logical presentation, business, data, and service layers as appropriate.
					\item
						nconsistent or poorly managed development processes. Consider performing an \acr{ALM} assessment, and make use of tried and tested development tools and methodologies.
					\item
						Lack of collaboration and communication between different groups involved in the application life-cycle. Consider establishing a development process integrated with tools to facilitate process work-flow, communication, and collaboration.
					\item
						Lack of design and coding standards. Consider establishing published guidelines for design and coding standards, and incorporating code reviews into your development process to ensure guidelines are followed.
					\item
						Existing (legacy) system demands can prevent both refactoring and progression toward a new platform or paradigm. Consider how you can create a migration path away from legacy technologies, and how to isolate applications from external dependencies. For example, implement the Gateway design pattern for integration with legacy systems.
				\end{itemize}
			
			\subsection{Interoperability}
				\paragraph{\gls{Interoperability} is the ability of a system or different systems to operate successfully by communicating and exchanging information with other external systems written and run by external parties. An interoperable system makes it easier to exchange and reuse information internally as well as externally. Communication protocols, interfaces, and data formats are the key considerations for interoperability. Standardization is also an important aspect to be considered when designing an interoperable system. The key issues for interoperability are:}
				
				\begin{itemize}
					\item
						Interaction with external or legacy systems that use different data formats. Consider how you can enable systems to interoperate, while evolving separately or even being replaced. For example, use orchestration with adaptors to connect with external or legacy systems and translate data between systems; or use a canonical data model to handle interaction with a large number of different data formats.
					\item
						Boundary blurring, which allows artifacts from one system to defuse into another. Consider how you can isolate systems by using service interfaces and/or mapping layers. For example, expose services using interfaces based on XML or standard types in order to support interoperability with other systems. Design components to be cohesive and have low coupling in order to maximize flexibility and facilitate replacement and reusability.
					\item
						Lack of adherence to standards. Be aware of the formal and de facto standards for the domain you are working within, and consider using one of them rather than creating something new and proprietary.
					
				\end{itemize}
			
			\subsection{Maintainability}
			
				\paragraph{\gls{Maintainability} is the ability of the system to undergo changes with a degree of ease. These changes could impact components, services, features, and interfaces when adding or changing the application’s functionality in order to fix errors, or to meet new business requirements. Maintainability can also affect the time it takes to restore the system to its operational status following a failure or removal from operation for an upgrade. Improving system maintainability can increase availability and reduce the effects of run-time defects. An application’s maintainability is often a function of its overall quality attributes but there a number of key issues that can directly affect maintainability:}
				
				\begin{itemize}
					\item
						Excessive dependencies between components and layers, and inappropriate coupling to concrete classes, prevents easy replacement, updates, and changes; and can cause changes to concrete classes to ripple through the entire system. Consider designing systems as well-defined layers, or areas of concern, that clearly delineate the system’s UI, business processes, and data access functionality. Consider implementing cross-layer dependencies by using abstractions (such as abstract classes or interfaces) rather than concrete classes, and minimize dependencies between components and layers.
					\item
						The use of direct communication prevents changes to the physical deployment of components and layers. Choose an appropriate communication model, format, and protocol. Consider designing a pluggable architecture that allows easy upgrades and maintenance, and improves testing opportunities, by designing interfaces that allow the use of plug-in modules or adapters to maximize flexibility and extensibility.
					\item
						Reliance on custom implementations of features such as authentication and authorization prevents reuse and hampers maintenance. To avoid this, use the built-in platform functions and features wherever possible.
					\item
						The logic code of components and segments is not cohesive, which makes them difficult to maintain and replace, and causes unnecessary dependencies on other components. Design components to be cohesive and have low coupling in order to maximize flexibility and facilitate replacement and reusability.
					\item
						The code base is large, unmanageable, fragile, or over complex; and refactoring is burdensome due to regression requirements. Consider designing systems as well defined layers, or areas of concern, that clearly delineate the system’s UI, business processes, and data access functionality. Consider how you will manage changes to business processes and dynamic business rules, perhaps by using a business workflow engine if the business process tends to change. Consider using business components to implement the rules if only the business rule values tend to change; or an external source such as a business rules engine if the business decision rules do tend to change.
					\item
						The existing code does not have an automated regression test suite. Invest in test automation as you build the system. This will pay off as a validation of the system’s functionality, and as documentation on what the various parts of the system do and how they work together.
					\item
						Lack of documentation may hinder usage, management, and future upgrades. Ensure that you provide documentation that, at minimum, explains the overall structure of the application.
				\end{itemize}
			
			\subsection{Manageability}
			
				\paragraph{\gls{Manageability} defines how easy it is for system administrators to manage the application, usually through sufficient and useful instrumentation exposed for use in monitoring systems and for debugging and performance tuning. Design your application to be easy to manage, by exposing sufficient and useful instrumentation for use in monitoring systems and for debugging and performance tuning. The key issues for manageability are:}
				
				\begin{itemize}
					\item
						Lack of health monitoring, tracing, and diagnostic information. Consider creating a health model that defines the significant state changes that can affect application performance, and use this model to specify management instrumentation requirements. Implement instrumentation, such as events and performance counters, that detects state changes, and expose these changes through standard systems such as Event Logs, Trace files, or \acr{WMI}. Capture and report sufficient information about errors and state changes in order to enable accurate monitoring, debugging, and management. Also, consider creating management packs that administrators can use in their monitoring environments to manage the application.
					\item
						Lack of runtime configurability. Consider how you can enable the system behavior to change based on operational environment requirements, such as infrastructure or deployment changes.
					\item
						Lack of troubleshooting tools. Consider including code to create a snapshot of the system’s state to use for troubleshooting, and including custom instrumentation that can be enabled to provide detailed operational and functional reports. Consider logging and auditing information that may be useful for maintenance and debugging, such as request details or module outputs and calls to other systems and services.
					\item
				\end{itemize}
			
			\subsection{Performance}
				
				\paragraph{\gls{Performance} is an indication of the responsiveness of a system to execute specific actions in a given time interval. It can be measured in terms of latency or throughput. Latency is the time taken to respond to any event. Throughput is the number of events that take place in a given amount of time. An application’s performance can directly affect its scalability, and lack of scalability can affect performance. Improving an application’s performance often improves its scalability by reducing the likelihood of contention for shared resources. Factors affecting system performance include the demand for a specific action and the system’s response to the demand. The key issues for performance are:}
				
				\begin{itemize}
					\item
						Increased client response time, reduced throughput, and server resource over utilization. Ensure that you structure the application in an appropriate way and deploy it onto a system or systems that provide sufficient resources. When communication must cross process or tier boundaries, consider using coarse-grained interfaces that require the minimum number of calls (preferably just one) to execute a specific task, and consider using asynchronous communication.
					\item
						Increased memory consumption, resulting in reduced performance, excessive cache misses (the inability to find the required data in the cache), and increased data store access. Ensure that you design an efficient and appropriate caching strategy.
					\item
						Increased database server processing, resulting in reduced throughput. Ensure that you choose effective types of transactions, locks, threading, and queuing approaches. Use efficient queries to minimize performance impact, and avoid fetching all of the data when only a portion is displayed. Failure to design for efficient database processing may incur unnecessary load on the database server, failure to meet performance objectives, and costs in excess of budget allocations.
					\item
						Increased network bandwidth consumption, resulting in delayed response times and increased load for client and server systems. Design high performance communication between tiers using the appropriate remote communication mechanism. Try to reduce the number of transitions across boundaries, and minimize the amount of data sent over the network. Batch work to reduce calls over the network.
				\end{itemize}
			
			\subsection{Reliability}
			
			
				\paragraph{\gls{Reliability} is the ability of a system to continue operating in the expected way over time. Reliability is measured as the probability that a system will not fail and that it will perform its intended function for a specified time interval. The key issues for reliability are:}
				
				
				\begin{itemize}
					\item
						The system crashes or becomes unresponsive. Identify ways to detect failures and automatically initiate a failover, or redirect load to a spare or backup system. Also, consider implementing code that uses alternative systems when it detects a specific number of failed requests to an existing system.
					\item
						Output is inconsistent. Implement instrumentation, such as events and performance counters, that detects poor performance or failures of requests sent to external systems, and expose information through standard systems such as Event Logs, Trace files, or WMI. Log performance and auditing information about calls made to other systems and services.
					\item
						The system fails due to unavailability of other externalities such as systems, networks, and databases. Identify ways to handle unreliable external systems, failed communications, and failed transactions. Consider how you can take the system offline but still queue pending requests. Implement store and forward or cached message-based communication systems that allow requests to be stored when the target system is unavailable, and replayed when it is online. Consider using Windows Message Queuing or BizTalk Server to provide a reliable once-only delivery mechanism for asynchronous requests.
					
				\end{itemize}
				
				
			\subsection{Reusability}
			
			
				\paragraph{\gls{Reusability} is the probability that a component will be used in other components or scenarios to add new functionality with little or no change. Reusability minimizes the duplication of components and the implementation time. Identifying the common attributes between various components is the first step in building small reusable components for use in a larger system. The key issues for reusability are:}
			
				\begin{itemize}
					\item
						The use of different code or components to achieve the same result in different places; for example, duplication of similar logic in multiple components, and duplication of similar logic in multiple layers or subsystems. Examine the application design to identify common functionality, and implement this functionality in separate components that you can reuse. Examine the application design to identify crosscutting concerns such as validation, logging, and authentication, and implement these functions as separate components.
					\item
						The use of multiple similar methods to implement tasks that have only slight variation. Instead, use parameters to vary the behavior of a single method.
					\item
						Using several systems to implement the same feature or function instead of sharing or reusing functionality in another system, across multiple systems, or across different subsystems within an application. Consider exposing functionality from components, layers, and subsystems through service interfaces that other layers and systems can use. Use platform agnostic data types and structures that can be accessed and understood on different platforms.
					
				\end{itemize}
				
				
			\subsection{Scalability}
			
				\paragraph{\gls{Scalability} is ability of a system to either handle increases in load without impact on the performance of the system, or the ability to be readily enlarged. There are two methods for improving scalability: scaling vertically (scale up), and scaling horizontally (scale out). To scale vertically, you add more resources such as CPU, memory, and disk to a single system. To scale horizontally, you add more machines to a farm that runs the application and shares the load. The key issues for scalability are:}
			
				\begin{itemize}
					\item
						Applications cannot handle increasing load. Consider how you can design layers and tiers for scalability, and how this affects the capability to scale up or scale out the application and the database when required. You may decide to locate logical layers on the same physical tier to reduce the number of servers required while maximizing load sharing and failover capabilities. Consider partitioning data across more than one database server to maximize scale-up opportunities and allow flexible location of data subsets. Avoid stateful components and subsystems where possible to reduce server affinity.
					\item
						Users incur delays in response and longer completion times. Consider how you will handle spikes in traffic and load. Consider implementing code that uses additional or alternative systems when it detects a predefined service load or a number of pending requests to an existing system.
					\item
						The system cannot queue excess work and process it during periods of reduced load. Implement store-and-forward or cached message-based communication systems that allow requests to be stored when the target system is unavailable, and replayed when it is online.
					
				\end{itemize}
				
			
			\subsection{Security}
				
				\paragraph{\gls{Security} is the capability of a system to reduce the chance of malicious or accidental actions outside of the designed usage affecting the system, and prevent disclosure or loss of information. Improving security can also increase the reliability of the system by reducing the chances of an attack succeeding and impairing system operation. Securing a system should protect assets and prevent unauthorized access to or modification of information. The factors affecting system security are confidentiality, integrity, and availability. The features used to secure systems are authentication, encryption, auditing, and logging. The key issues for security are:}
				
				\begin{itemize}
					\item
						Spoofing of user identity. Use authentication and authorization to prevent spoofing of user identity. Identify trust boundaries, and authenticate and authorize users crossing a trust boundary.
					\item
						Damage caused by malicious input such as \gls{MYSQL} injection and cross-site scripting. Protect against such damage by ensuring that you validate all input for length, range, format, and type using the constrain, reject, and sanitize principles. Encode all output you display to users.
					\item
						Data tampering. Partition the site into anonymous, identified, and authenticated users and use application instrumentation to log and expose behavior that can be monitored. Also use secured transport channels, and encrypt and sign sensitive data sent across the network.
					\item
						Repudiation of user actions. Use instrumentation to audit and log all user interaction for application critical operations.
					\item
						Information disclosure and loss of sensitive data. Design all aspects of the application to prevent access to or exposure of sensitive system and application information.
					\item
						Interruption of service due to  \acr{DoS} attacks. Consider reducing session timeouts and implementing code or hardware to detect and mitigate such attacks.
					
				\end{itemize}
				
			
			\subsection{Supportability}
			
				\paragraph{\gls{Supportability} is the ability of the system to provide information helpful for identifying and resolving issues when it fails to work correctly. The key issues for supportability are:}
			
				\begin{itemize}
					\item
						ack of diagnostic information. Identify how you will monitor system activity and performance. Consider a system monitoring application, such as Microsoft System Center.
					\item
						Lack of troubleshooting tools. Consider including code to create a snapshot of the system’s state to use for troubleshooting, and including custom instrumentation that can be enabled to provide detailed operational and functional reports. Consider logging and auditing information that may be useful for maintenance and debugging, such as request details or module outputs and calls to other systems and services.
					\item
						Lack of tracing ability. Use common components to provide tracing support in code, perhaps though \acr{AOP} techniques or dependency injection. Enable tracing in Web applications in order to troubleshoot errors.
					\item
						Lack of health monitoring. Consider creating a health model that defines the significant state changes that can affect application performance, and use this model to specify management instrumentation requirements. Implement instrumentation, such as events and performance counters, that detects state changes, and expose these changes through standard systems such as Event Logs, Trace files, or \acr{WMI}.
					\item
						Capture and report sufficient information about errors and state changes in order to enable accurate monitoring, debugging, and management. Also, consider creating management packs that administrators can use in their monitoring environments to manage the application.
					
				\end{itemize}			
			
				\subsection{Testability}
				
					\paragraph{\gls{Testability} is a measure of how well system or components allow you to create test criteria and execute tests to determine if the criteria are met. Testability allows faults in a system to be isolated in a timely and effective manner. The key issues for testability are:}
					
					\begin{itemize}
						\item
							Complex applications with many processing permutations are not tested consistently, perhaps because automated or granular testing cannot be performed if the application has a monolithic design. Design systems to be modular to support testing. Provide instrumentation or implement probes for testing, mechanisms to debug output, and ways to specify inputs easily. Design components that have high cohesion and low coupling to allow testability of components in isolation from the rest of the system.
						\item
							Lack of test planning. Start testing early during the development life cycle. Use mock objects during testing, and construct simple, structured test solutions.
						\item
							Poor test coverage, for both manual and automated tests. Consider how you can automate user interaction tests, and how you can maximize test and code coverage.
						\item
							Input and output inconsistencies; for the same input, the output is not the same and the output does not fully cover the output domain even when all known variations of input are provided. Consider how to make it easy to specify and understand system inputs and outputs to facilitate the construction of test cases.
						
					\end{itemize}
					
					
				\subsection{User Experience / Usability}
				
					\paragraph{\gls{User Experience / Usability} The application interfaces must be designed with the user and consumer in mind so that they are intuitive to use, can be localized and globalized, provide access for disabled users, and provide a good overall user experience. The key issues for user experience and usability are:}
					
					\begin{itemize}
						\item
							too much interaction (an excessive number of clicks) required for a task. Ensure you design the screen and input flows and user interaction patterns to maximize ease of use.
						\item
							Incorrect flow of steps in multistep interfaces. Consider incorporating workflows where appropriate to simplify multistep operations.
						\item
							Data elements and controls are poorly grouped. Choose appropriate control types (such as option groups and check boxes) and lay out controls and content using the accepted UI design patterns.
						\item
							Feedback to the user is poor, especially for errors and exceptions, and the application is unresponsive. Consider implementing technologies and techniques that provide maximum user interactivity, such as \acr{AJAX} in Web pages and client-side input validation. Use asynchronous techniques for background tasks, and tasks such as populating controls or performing long-running tasks.
						
					\end{itemize}
\end{document}
\documentclass[../Psychological_system_web_application.tex]{subfiles}



\begin{document}
	
		%%%%%%%%%%%%%%%%%%%%%%%%%%%%%%%%%%%%%%%%%%%%%%%%%%%%%%%%%%%%%%%%%%%%%%%%%%%%%%%%%%%%%%%%%%%%%%%%%%%%%%%%%%	
		%Motivation to psychology science and catagroies of psychologists
		\section{MOTIVATION}
		
			\paragraph{The \gls{psychology} is the science, It aims to understand the \gls{behavior} of others and gather information about the way the brain works in order to better serve humanity, Psychology has four main goals:}
				\begin{enumerate}
					\item
						\textsc{\textbf{\color{red} Describe behavior:}}\\
							first goal of \gls{psychology} is to describe \gls{behavior}. description involves naming and classifying \gls{behavior}. This description is based on careful, systematic procedure in contrast to haphazard description of common sense. Description is very important in that it makes you clear about what the \gls{phenomena} under study. only after we described the \gls{behavior} or phenomenon clearly we can move to the other goals.
						\item
							\textsc{\textbf{\color{red} Understand or explain behavior:}}\\
								The second goal of psychology becomes explaining the \gls{behavior} or phenomenon that was described. \gls{psychologist}s who are concern this goal try to find out why such behavior occur. they take help of existing theories and knowledge to explain or understand behavior. in some cases if there are no theorizes or researches that can explain such behavior psychologists make tentative statements and try to test such hypothesis.
						\item
							\textsc{\textbf{\color{red} Predict the behavior:}}\\
								Another important goal for \gls{psychologist}s is to forecast future event. By carefully analyzing the relationship between different variables, psychologists can accurately predict what will be the relation in future between them. prediction helps in modifying the behavior. it is facilitated by understanding of the relationship.
						\item
							\textsc{\textbf{\color{red} Control or modify behaviors:}}\\
								The fourth goal of \gls{psychology} is to control, modify or change the existing behavior. the \gls{behavior}s that need to be corrected are modified through the help of psychological techniques. Only \gls{psychologist}s who work in applied are of psychology are concerned with controlling the behaviors. 
				\end{enumerate}
			
			\paragraph{A \gls{psychologist} interested in gender and women's issues teaches at the communty college and workes with her college and community to eliminate sexual harassement, and the researcher works at Educational Testing service to ferret out possible cultural bias in psychological tests.}
			
			\textsl{\textbf{Contemporary Approaches to \gls{psychology}}}
				\begin{itemize}
					\item
						\textsf{\textbf{\color{red} The Behavioral Approach:}}\\
							\emph{stresses that behavior is determined not only by environmental conditions but also by how thoughts modify the impact of environment on behavior.}
					\item
						\textsf{\textbf{\color{red} Psychosomatic approach:}}\\
							\emph{Emphasizes the unconscious aspects of the mind, conflict between biological instincts and society's demands and early family experiences.}
					\item
						\textsf{\textbf{\color{red} Cognitive Approach:}}\\
							\emph{focuses on the \gls{mental processes} involved in knowing how we direct our attention, perceive, remember, think, and solve problems.}
					\item
						\textsf{\textbf{\color{red}Behavioral neuroscience Approach:}}\\
							\emph{views understanding the brain and nervous system as central to understanding behavior, thought, and emotion.}
					\item
						\textsf{\textbf{\color{red} Evaloutionary Psychology Approach:}}\\
							\emph{Emphasizes the importance of functional purpose and adaptation in explaining why behaviors are formed, are modified, and survive.}
					\item
						\textsf{\textbf{\color{red} Sociocultural Approach:}}\\
							\emph{Emphasizes social and cultural influences on behavior.}
				\end{itemize}			
			
%%%%%%%%%%%%%%%%%%%%%%%%%%%%%%%%%%%%%%%%%%%%%%%%%%%%%%%%%%%%%%%%%%%%%%%%%%%%%%%%%%%%%%%%%%%%%%%%%%%%%%%%%%
		%Motivetion to chapter 0
		\paragraph{  In this chapter we will introduce an overview at all flowing chapters and look at the purpose of documentation , then introduce the scope of document, and introduce some definitions' terms and acronyms that will use in the document, then mention the References that import from it some information that serve this document, finally we will Describe in generally the contents of this document}
		
			
%%%%%%%%%%%%%%%%%%%%%%%%%%%%%%%%%%%%%%%%%%%%%%%%%%%%%%%%%%%%%%%%%%%%%%%%%%%%%%%%%%%%%%%%%%%%%%%%%%%%%%%%%%
		\section{PURPOSE}
			
			\paragraph{ main purpose from this document build Online System to manage daily work on the psychological center and illustrate and explain The system Of psychological center and all contents that included in it, and introduce or explore the problems and statues in the psychological center, and knowing the constraints, and introduce some efficient solutions that solve the specific problem, and representing the functional and nonfunctional of system.}
			
			
		
%%%%%%%%%%%%%%%%%%%%%%%%%%%%%%%%%%%%%%%%%%%%%%%%%%%%%%%%%%%%%%%%%%%%%%%%%%%%%%%%%%%%%%%%%%%%%%%%%%%%%%%%%%
		\section{INTENDED AUDIENCE AND READING SUGGESTIONS}
			\paragraph*{we will introduce in }
				\begin{itemize}
				\item \textbf{chapter one}\\
					in this chapter will introduce general description of the project and product functions and constraints 
				\item \textbf{chapter two}\\
					will introduce the functional, performance, interface  and explain the database
				\item \textbf{chapter three}\\
					will introduce the results and the features of the system  
				\item \textbf{chapter four}\\
					will introduce the Appendices 
				\item \textbf{chapter five}\\
					will introduce the index
			\end{itemize}			
			\paragraph*{This Project is a prototype for intelligent Management system and it restricted to create the daily tables of appointments of sessions and courses of the training department, send official confirmation message to clients and \gls{psychologist}s to confirm the reservation of the flow session, compute the total income and total outcome, then calculate the revenue in gaily, monthly, annually, calculate the salaries of employees and output ratio of psychologists then outcome the total amount per day, and finally generate some reports to help head Managers to make decisions in the fast time, we will introduce this in the flowing sections in details}
			%%%%%%%%%%%%%%%%%%%%%%%%%%%%%%%%%%%%%%%%%%%%%%%%%%%%%%%%%%%%%%%%%%%%%%%%%%%%%%%%%%%%%%%%%%%%%%%%%%%%%%%%%%%%%%%%%%%%%%%%%%%%%%%%%%%%%%%%%%%%%%%%%%%%%%%%%%%%%%%%%%%%%%%%%%%%%%%%%%%%%%%%%%%%%%%%%%
		
		\section{PROJECT SCOPE}
		
			\paragraph*{ scope of this document is helping the employees in the psychological center and improve the methods that them was used to it, and helping the \gls{psychologist}s to work with more efficient and saving the time of them, and helping the head managers to take the correct decisions in the appropriate time, we will use some devices and techniques like fingerprint device to manage the attendance on the center }
			\paragraph{ }
			
%%%%%%%%%%%%%%%%%%%%%%%%%%%%%%%%%%%%%%%%%%%%%%%%%%%%%%%%%%%%%%%%%%%%%%%%%%%%%%%%%%%%%%%%%%%%%%%%%%%%%%%%%%
\end{document}